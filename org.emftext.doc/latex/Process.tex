\chapter{Development Process}

Creating parsers, printers and editors with EMFText is easy! It involves some
necessary steps which are roughly depicted in the activity diagram below.

% TODO [[Image:emftext_process.png|center|thumb|800px]]

The starting point usually is the '''Ecore-based model''' serving as an abstract 
syntax and as a skeleton for concrete syntax. The model can be a graph or tree 
definition. Although Ecore models are specified as XML files, it is recommended 
to use the Ecore model editor or an other graphical editor. Additionally, a unique 
namespace (property NS URI in the .ecore file) for the models as well as a package 
name needs to be specified. To enable EMFText to use models at runtime, a model 
plug-in must be generated. For this, EMF provides a \emph{generation
metamodel} (GenModel) allowing to enrich Ecore models with specific information
for code generation. The EMF Model wizard facilitates the creation of GenModels.

The next step is to define the base package name which will be the common prefix 
for all subsequently generated Java packages. After finishing configuration, the 
\emph{EMF model plug-in} is ready to be generated. EMFText will later use
these classes to construct a model instance.

Since our metamodel is now defined, we can start specifying our \emph{concrete
syntax}. As a starting point, EMFText provides a syntax generator that can
automatically create a concrete-syntax specification using HUTN (Human-Useable Textual Notation) from the 
metamodel. To manually specify the concrete syntax, a textual specification has to 
be written, which has the ending \texttt{cs} (from \texttt{c}oncrete
\texttt{s}yntax). It consists of four sections: In the first part, a unique
name refers to the syntax as a specific resource type. Furthermore, the metamodel, whose textual syntax shall be defined, has 
to be referenced by its  unique namespace URI and a start symbol must be selected from 
the model elements. In the second part, productions from other concrete syntax 
specifications can be imported. This is especially useful if the metamodel is 
compositionally structured, e.g. reuses large parts from other models which have an 
already specified concrete syntax.  The third part allows for specifying token types as
 it is usual for parser generators, but in contrast, EMFText also allows to leave them 
 out: If no token definitions are given, default rules are used. Another specialty for 
 token definitions are optional pre- and suffixes which are transparently removed (after 
 parsing) and added (before printing)  by the generated token resolvers. And last but 
 not least, EBNF-like productions have to be defined for each non-abstract model element 
 reachable from the start symbol. They have to be defined with respect to the attributes 
 and references of the model element and thus do not contain nonterminals in the classic 
 sense.

The context menu on concrete syntax specification files offers an item to generate the 
actual \emph{resource plug-in} which contains the parser, printer and the
editor for the language. In an optional last step, the generated token resolvers and the printer can be 
tailored to specific needs.
