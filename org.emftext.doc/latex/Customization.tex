\chapter{DSL Customization}

\section{Customization Techniques}

\subsection{Overriding Generated Classes}

\subsection{Using Generated Extension Points}



\section{Concrete Customizations}

\subsection{Customizing Token Resolving}
\label{sec:cust_token_resolving}

\subsection{Customizing Reference Resolving}
\label{sec:cust_reference_resolving}

\subsection{Implementing Post Processors}

\subsection{Implementing Quick Fixes}

\subsection{Implementing Builders}

To implement a custom builder for your \DSL, basically set the code generation
option \texttt{overrideBuilder} to \texttt{false}:

\begin{lstlisting}
OPTIONS {
    overrideBuilder = "false";
}
\end{lstlisting}

After regenerating the resource plug-ins (see
Sect.~\ref{sec:process_generating}), you will find a new class
\texttt{XyzBuilder} in the \texttt{src} folder of the generated resource 
plug-in (assuming the file extension of your \DSL is \texttt{xyz}. If you
face compilation errors, make sure to the delete the \texttt{XyzBuilder} 
class from the \texttt{src-gen} folder.

The generated builder class contains two methods---\texttt{isBuildingNeeded()} and
\texttt{build()}. The first one is called to let the builder decide, which
resources need to be included in the build process. The default implementation
returns \texttt{false} to avoid unnecessary loading of resources. To include all
textual resources that contain models of your \DSL, change the method to return
\texttt{true}.

The second method is called whenever the content of a resource changes. You can
implement arbitrary behavior here. Usually, builders create some kind of derived
artifact, for example a transformed or compiled version of the \DSL model. Since
\texttt{build()} retrieves the resource as method parameter, you can easily
access the contents of the resource. To save the derived artifact it is good
practice to use the URI of the original resource to derive a new URI. This can
for example be done by removing segments and adding new ones.

The following listing shows a simple builder, which copies the contents of the
resource to a new resource without making any changes.

\begin{lstlisting}
import java.io.IOException;
import java.util.Collection;

import org.eclipse.core.runtime.IProgressMonitor;
import org.eclipse.core.runtime.IStatus;
import org.eclipse.core.runtime.Status;
import org.eclipse.emf.common.util.EList;
import org.eclipse.emf.common.util.URI;
import org.eclipse.emf.ecore.EObject;
import org.eclipse.emf.ecore.resource.Resource;
import org.eclipse.emf.ecore.util.EcoreUtil;
import org.emftext.language.xyz.resource.xyz.IXyzBuilder;

public class XyzBuilder implements IXyzBuilder {
  
  public boolean isBuildingNeeded(URI uri) {
    return true;
  }
  
  public IStatus build(XyzResource resource, IProgressMonitor monitor) {
    // get contents and create copy 
    EList<EObject> contents = resource.getContents();
    Collection<EObject> contentsCopy = EcoreUtil.copyAll(contents);
    
    // create new resource with different name
    URI newUri = URI.createURI("copy.xyz").resolve(resource.getURI());
    Resource newResource = resource.getResourceSet().createResource(newUri);
    // add copy of original content to new resource
    newResource.getContents().addAll(contentsCopy);
    // save new resource
    try {
      newResource.save(null);
    } catch (IOException e) {
      // handle exception
    }
    return Status.OK_STATUS;
  }
}
\end{lstlisting}

Alternatively, you can also register builders for your \DSL in other plug-ins.

\subsection{Implementing Interpreters}

\subsection{Customizing Text Hovers}

To implement custom text hovers for your \DSL, basically set the code
generation option \texttt{overrideHoverTextProvider} to \texttt{false}:

\begin{lstlisting}
OPTIONS {
    overrideHoverTextProvider = "false";
}
\end{lstlisting}

After regenerating the resource plug-ins (see
Sect.~\ref{sec:process_generating}), a new class
\texttt{XyzHoverTextProvider} can be found in the \texttt{src} folder of the
generated resource UI plug-in (assuming the file extension of your \DSL is \texttt{xyz}.
If you face compilation errors, make sure to the delete the \texttt{XyzHoverTextProvider} 
class from the \texttt{src-gen} folder.

The generated hover text provider class contains one
method---\texttt{getHoverText()}. The default implementation of this
method delegates calls to a default provider. To customize the
hover text you can inspect the \texttt{EObject} passed to the method and return arbitrary HTML code. The
following listing shows a simple customized provider, which returns the type of
the \texttt{EObject}.

\begin{lstlisting}
import org.eclipse.emf.ecore.EObject;
import org.emftext.language.xyz.resource.xyz.IXyzHoverTextProvider;

public class XyzHoverTextProvider implements IXyzHoverTextProvider {
	
  public String getHoverText(EObject object) {
    return "An object of type " + object.eClass().getName();
  }
}
\end{lstlisting}

\subsection{Customizing Code Completion Proposals}

