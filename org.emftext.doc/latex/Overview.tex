\chapter{Overview}
EMFText is a tool for defining textual syntax for Ecore-based metamodels. It 
enables developers to define their own textual languages---be it domain
specific languages (e.g., a language for describing forms) or general purpose 
languages (e.g., Java)---and generates accompanying tool support for these 
languages. It provides a domain specific language (DSL) for syntax 
specification from which it generates a full-fledged Eclipse editor and 
components to load and store model instances.

To give a quick overview, some of the most compelling features of EMFText are 
outlined in the following.

\section{Generation features}

\begin{description}

  \item[Generation of independent code]
        The code that is generated by EMFText does not contain dependencies
        to EMFText and is fully customizable. This implies that generated 
        language tooling can be deployed in environments where EMFText is not 
        available and that future compatibility issues are completely avoided.

  \item[Generation of default syntax]
        With EMFText, an initial syntax for the textual DSL can be generated 
        in one step for any Ecore-based metamodel. This syntax conforms to the 
        HUTN standard. From this initial version of generated specification the 
        syntax can be further tailored towards specific needs (cf.~Sect.~\ref{sec:process_specification}).

\end{description}


\section{Specification features}

\begin{description}

  \item[Simple and precise syntax specification]
        EMFText comes with a simple but rich syntax specification language---the
        \emph{Concrete Syntax Specification Language (CS)}. It is based on EBNF 
        and follows the concept of \emph{convention over configuration}. This 
        allows for very compact and intuitive syntax specifications, but still 
        supports tweaking specifics where needed (cf.~Chap.~\ref{chap:cs}).

  \item[Modular specification]
        EMFText provides an import mechanism that not only supports specification 
        of a single text syntax for multiple related Ecore models, but also allows 
        for modularization and extension of CS specifications (cf.~Sect.~\ref{sec:cs_import}).

  \item[Default reference resolving mechanisms]
        A default name resolution mechanism for models with globally unique 
        names is available out of the box for any syntax. More complex resolution 
        mechanisms can be realized by implementing generated resolving methods 
        through which also inter-model references can be established 
        (cf.~Sect.~\ref{sec:cust_reference_resolving}).

  \item[Comprehensive syntax analysis]
        A number of analyses of CS specifications inform the developer about 
        potential errors in the syntax---like missing syntax for certain 
        metaclasses (cf.~Appx.~\ref{app:warnings}).

\end{description}


\section{Editor features}

EMFText editors provide many advanced features like code completion, 
customizable syntax and occurrence highlighting, advanced bracket handling,
hyperlinks and text hovers, an outline view and instant error reporting.


\section{Other features}

\begin{description}

  \item[Support for ANT]
        Dedicated ANT task are provided to all the generation of text syntax 
        plug-ins in build scripts (cf.~Sect.~\ref{sec:process_generating_ant}).

  \item[Generation of builder and interpreter stubs]
        EMFText generates a builder stub that can be used to process 
        model instances on changes and to automatically produce derived 
        resources when needed (cf.~Sect.~\ref{sec:cust_builders}). 
        Similarly, interpreters are used to execute model instances 
        (cf.~Sect.~\ref{sec:cust_interpreters}).
        
\end{description}