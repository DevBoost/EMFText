\chapter{Overview}
EMFText is a tool for defining textual syntax for Ecore-based metamodels. It 
enables developers to define their own textual languages---be it domain
specific languages (e.g., a language for describing forms) or general purpose 
languages (e.g., Java)---and generates accompanying tool support for these 
languages. It provides a domain specific language (DSL) for syntax 
specification from which it generates a full-fledged Eclipse editor and 
components to load and store model instances.

To give a quick overview, some of the most compelling features of EMFText are 
outlined in the following.

\section{Generation features}
EMFText uses a generative approach where all artifacts that form the tooling for
a textual language are generated. This includes a parser for loading textual 
models, a printer for storing model instances and the editor with all its
customizable components.

\begin{description}

  \item[Generation of independent code]
        The code that is generated by EMFText does not contain dependencies
        to EMFText and is fully customizable. This implies that generated 
        language tooling can be deployed in environments where EMFText is not 
        available and that future compatibility issues are completely avoided.

  \item[Generation of default syntaxes]
        With EMFText, initial syntaxes for the textual DSL can be generated 
        in one step for any Ecore-based metamodel. One can generate a syntax 
        that conforms to the HUTN standard~\cite{HUTN}, a Java-like syntax, or
        a custom syntax configured by using the custom syntax wizard. In all 
        cases, the initial, generated specification of the syntax can be further 
        tailored towards specific needs 
        (cf.~Section~\ref{sec:process_specification}).
        
  \item[Highly customizable code generation]
        EMFText provides many options for tailoring its code generation process
        to specific needs. For example, manually modified code can be preserved
        by disabling its generation or custom license headers can be provided 
        if needed (cf.~Appendix~\ref{app:options}).

\end{description}


\section{Specification features}

EMFText comes with a simple but rich syntax specification language---the
\emph{Concrete Syntax Specification Language (CS)}. It is based on EBNF 
and follows the concept of \emph{convention over configuration}. This 
allows for very compact and intuitive syntax specifications, but still 
supports tweaking specifics where needed (cf.~Chapter~\ref{chap:cs}).

\begin{description}

  \item[Modular specification]
        EMFText provides an import mechanism that not only supports specification 
        of a single text syntax for multiple related Ecore models, but also allows 
        for modularization and extension of CS specifications (cf.~Section~\ref{sec:cs_import}).

  \item[Default reference resolving mechanisms]
        A default name resolution mechanism for models with globally unique 
        names is available out of the box for any syntax. Also, external
        references are resolved automatically, if URIs are used to point to the
        referenced elements. More complex resolution mechanisms can be realized
        by implementing generated resolving methods 
        (cf.~Section~\ref{sec:cust_reference_resolving}).

  \item[Comprehensive syntax analysis]
        A number of analyses of CS specifications inform the developer about 
        potential errors in the syntax---like missing syntax for certain 
        metaclasses (cf.~Appendix~\ref{app:warnings}).

\end{description}


\section{Editor features}

Editors generated by EMFText provide many advanced features that are known from,
e.g., the Eclipse Java editor. This includes code completion (with customizable 
completion proposals cf.~Section~\ref{sec:cust_reference_resolving} 
and Section~\ref{sec:cust_code_completion}), customizable syntax and occurrence 
highlighting via preference pages, advanced bracket handling, code folding, 
hyperlinks and text hovers for quick navigation, an outline view and instant error reporting.


\section{Other features}

EMFText provides numerous other interesting features, some of them outlined below.

\begin{description}

  \item[ANT support]
        Dedicated ANT tasks are provided to allow the generation of text syntax 
        plug-ins in build scripts (cf.~Section~\ref{sec:process_generating_ant}).
        
  \item[Support for post processors]
        By default, registered post processors are called by the 
        tooling after parsing. These post processors can be customized to
        check consistency of models or perform necessary modifications after
        parsing (cf.~Section~\ref{sec:cust_post_processors}).

  \item[Generation of builder stubs]
        EMFText generates a builder stub that can be used to process 
        model instances on changes and to automatically produce derived 
        resources when needed (cf.~Section~\ref{sec:cust_builders}).
        
  \item[Generation of interpreter stubs]
        Similarly, interpreters are used to execute model instances 
        (cf.~Section~\ref{sec:cust_interpreters}).
  
  \item[Quick fixes]
        Quick fixes provide actions that can automatically solve problems found during
        analysis of model instances. EMFText provides means to attach quick fixes to
        reported problems which then can be fixed by the developer in a convenient way 
        (cf.~Section~\ref{sec:cust_quick_fixes}).
      
\end{description}