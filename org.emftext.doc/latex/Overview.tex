\chapter{Overview}

EMFText is a tool which allows users to define a plain textual syntax for their
Ecore based metamodel and to generate components to load, edit and store model 
instances. The syntax is specified by a so called concrete syntax specification 
which are usually stored as files with the suffix \texttt{.cs}. A \texttt{cs}
specification is directly related to one ore more Ecore metamodel(s) whose structure implicitly 
pretends a ''grammar skeleton''. The following figure gives an overview on how the 
generator part of EMFText works and what components it actually generates.

% TODO [[Image:emftext_schema.gif]]

Through combining metamodel and \texttt{cs} specification, EMFText derives a context-free 
grammar and exports it as an \ANTLR parser specification. This specification
contains annotated semantic actions which cover the largest part of metamodel 
instantiation. EMFText then transparently delegates
parser\footnote{http://en.wikipedia.org/wiki/Parser} and 
lexer\footnote{http://en.wikipedia.org/wiki/Lexical\_analysis} generation to
ANTLR by passing the generated grammar file. Since ANTLR does not cover the whole class of 
context-free grammars we can not guarantee the generation of a working parser 
for arbitrary cases. However, in most cases generation should be sufficient.

While parsers are used to load model instances from textual representations a 
printer is needed to do the inverse, e.g. to print an in-memory instance of the 
metamodel back to a textual representation. The printed results can then 
again be parsed by the corresponding parser. Both instances (printed and
loaded) should be equal. Furthermore, printers should produce a formatted and human-readable output. 
The EMFText built-in printer generator tries to achieve these goals by interpreting 
the \texttt{cs} file and the derived grammar. \texttt{cs} specifications can be enriched by 
special operators to indicate that on a specific position white-spaces or newlines 
have to be printed. Additionally, it uses information about literals (e.g. keywords) 
in defined languages which are removed from model instances. As for parser generation, 
EMFText does not guarantee that printer generation works for arbitrary cases but 
mostly it is be a convenient solution.

EMFText also generates a set of resolvers. Resolvers convert parsed token
strings to an adequate representation in the metamodel instance.

TokenResolvers implement a mapping from the string value of a specific token to a native 
Java type (e.g., boolean, int, String etc.). In the standard implementation
TokenResolvers can automatically remove and add (printing) pre- and suffixes. The conversion to native 
java data types is done by delegation to the corresponding Java type conversion
functions. For example, \texttt{Integer.parseInt("42")} results in an int valued
\texttt{42}. Since this behavior is only desired for concrete syntaxes
mirroring exactly (or at least partially) the Java syntax for primitive types, users are 
expected to implement more adequate mapping functions as needed.  

Resolvements depending on context are meant to be realised by implementing
ReferenceResolvers. For these only stubs are generated. While TokenResolvers are directly invoked by parsers, 
ReferenceResolvers are triggered on demand later by EMF's proxy resolution mechanism. An 
additional feature is the evaluation of eventually annotated \OCL constraints.
With \OCL, consistency conditions can be declared on the metamodel to further
improve quality of EMFText based developments.
