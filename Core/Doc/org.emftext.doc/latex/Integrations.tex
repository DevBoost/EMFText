\subsection{Integration with other Framework}

\subsection{Integration with the EMF Edit Framework}

The DSL tooling that is generated by EMFText is tightly integrated with the EMF
Edit Framework. Supposing you have generated the EMF Edit code for your Ecore 
model, the provider classes will be used to obtain icons and labels for the 
outline view. Also, the icons will be shown whenever appropriate (e.g., when
presenting code completion proposals). This integration with the EMF Edit 
Framework ensures consistent presentation of DSLs across different kinds of
editors.

\subsection{Integration with the EMF Validation Framework}

EMFText does also integrate your DSL tooling with the EMF Validation Framework.
By default, constraints that are registered using the Validation Framework are
checked by in the generated editor and shown as errors or warnings. Since, the
EMF Validation Framework differentiates between live and batch constraints,
generated editors handle these two types differently. When resources are loaded
for the first time (e.g., when an editor is opened), both constraint types are
evaluated. While editing DSL documents, only live constraints are evaluated.
Batch constraints are re-evaluated when documents are saved.

To ensure correct behavior of the constraint evaluation, make sure that the 
\texttt{constraintProvider} extension is defined to be mixed mode (i.e., 
attribute \texttt{mode} must not be set in the \texttt{plugin.xml} file). 
Otherwise, constraints will be evaluated only for one of the two modes, which 
will cause confusing behavior. For example, error markers for live constraints
might disappear when resources are saved. Other than that, constraints can be
regularly registered as defined by the extension points of the EMF Validation 
Framework.

Examples of integrating with the EMF Validation Framework can be found in
the DSLs HEDL\footnote{\url{http://www.emftext.org/language/hedl}} and 
Functions\footnote{\url{http://www.emftext.org/language/functions}}. 
